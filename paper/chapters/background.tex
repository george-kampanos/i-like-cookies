\section{Related Work}
Studies in this area have mainly focused on the prevalence of cookie banners, the type of privacy options they offer, and whether they comply with the law.

In 2018, The Norwegian Consumer Council reviewed whether user interfaces of cookie notices and privacy settings provided by Google, Facebook and Microsoft Windows 10 discourage users from making privacy-aware choices~\cite{council2018deceived}. They found that all three companies offer default settings that are considered privacy intrusive, and that the cookie notices contain misleading wording while privacy-friendly options require multiple steps to find. 
They noted that Google and Facebook \say{threaten users with loss of functionality or deletion of the user account} unless they agree to those privacy-intrusive settings.

A number of studies in this area focus on providing a big picture across the world or Europe. 
Habib et al. conducted a 150-website analysis in 2018--19 and found that privacy options are frequent within their sample with 89\% websites with targeted advertising offering a way to opt-out~\cite{habib2019empirical}. However, they observed that, when visited from the US, only 28 out of 150 websites they considered displayed a cookie banner with only 5 of them offering a means to opt out. 
Degeling et al.'s study of 6,759 websites across the EU found adoption of cookie banners across the EU go up from 46.1\% before the GDPR to 62.1\% afterwards~\cite{degeling2018we}. 
Utz et al. carried a manual inspection of 1,000 popular websites in the EU and observed that 27.8\% provide no options, 68.0\% allow confirmation only, while only 3.2\% give a binary accept/reject choice~\cite{utz2019informed}. 
Another study of top 100 websites in each EU country by van Eijk et al. found 52\% of UK and 29\% of Greek websites implementing a cookie banner~\cite{eijk2019impact}. 

Two recent studies have looked at whether cookie banners provided by Content Management Platforms (CMPs) adhere to EU regulations. 
In a study published in 2019, Matte, Bielova and Santos surveyed 1,427 European websites from which they found that 141 websites registered an affirmative consent before the user had performed any actions and 38 websites offered no \say{opt-out} option at all~\cite{matte2020cookie}. The authors observed that at least 50\% of the websites in their dataset had pre-selected privacy options and at least 27 websites did not respect the user’s choice even though they declined to be tracked by cookies. 
In a study published in 2020 considering 680 websites, Nouwens et al. found that 32\% of them assumed \say{implicit consent} (agreeing without having any other option)~\cite{nouwens2020dark}, which make those websites not-compliant with GDPR. They also found that only 13\% of websites had a \say{reject all} button which almost always required additional clicks to be seen by a user.

Taking a closer look at the sample sizes of the studies that focus on providing a big picture, we have Habib et al.'s study of 150 websites worldwide (50 from each group of high, middle, and low popularity)~\cite{habib2019empirical}, Degeling et al.'s sample of 6,759 websites including 463 UK and 443 Greek ones~\cite{degeling2018we}, Utz et al.'s 1,000 randomly chosen from 500 top-ranking in each EU country~\cite{utz2019informed}, and van Eijk et al.'s sample of top 100 in each EU country~\cite{eijk2019impact}. 
Similarly for the studies that limit their attention to websites with CMP-provided banners, these include Nouwens et al.'s study of 680 such websites in the UK~\cite{nouwens2020dark} and Matte, Bielova, and Santos's investigation of 1,426 websites including 149 \texttt{.uk} and 53 \texttt{.gr} websites~\cite{matte2020cookie}. 
Although such studies provide valuable insight, none goes beyond 700 websites in the two countries we consider. This is understandable due to the complexity of automating such studies and that the goal of the said studies was to focus on a global view or on CMPs, and not on the comprehensive landscape in specific countries. 
This opens a natural question whether similar trends can be seen if the scale of the sample sizes considered are increased. Indeed, it is not clear whether characteristics observed in high-traffic websites remain similar if low-traffic ones are considered. 
To answer this question, we focused on two specific countries: the UK and Greece, but scaled up the sample size nearly ten-fold by automating our scraping and analysis, allowing us to expand our research to more than 14,000 UK and 3,000 Greek websites. 

While we developed purpose-built software, this study relied heavily upon existing software. We extended OpenWPM, an open-source web privacy measurement tool developed by Englehardt and Narayanan in 2016 to scrape and collect data~\cite{englehardt2016census}. OpenWPM allows researchers to detect and measure the use of third-party cookies (TPs), cookie synchronisation, as well as fingerprinting techniques. We modified OpenWPM to be able to recognise and store the website's cookie banner if one exists. 