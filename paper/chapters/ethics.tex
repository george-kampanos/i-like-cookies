\documentclass[../main.tex]{subfiles}

\begin{document}

\begin{itemize}
    \setlength\itemsep{1em}

    \item \textbf{Aim}: This project aims to determine the number of privacy options that a visitor is given when visiting Greek \& UK websites. 

    \item \textbf{Methodology}: OpenWPM is an open-source web privacy measurement framework that allows users to collect privacy information from websites and is already been used in several other studies. Using OpenWPM, a large number of websites can be crawled and have their Cookie Banners detected and stored for further analysis (e.g. how many websites offer an \say{opt-out from third-party tracking} option).

    \item \textbf{Participants}: There are no participants in this study.

    \item \textbf{Data collection}: Only the HTML from the cookie/privacy notices is collected. This information cannot be used to deteriorate the security of that website (e.g no functional code is collected to identify bugs). However, the robots.txt file is respected and websites that \say{Deny} crawlers or have no robots.txt at all are not scraped. Furthermore, this project makes best efforts to respect the \say{Terms \& Conditions} of the websites in our list by ensuring that they do not contain phrases such as \say{for personal use only}. If such exclusionary terms are found, then the website is not crawled. Specific details of the robots.txt and Terms of Service compliance are discussed in detail in Chapters \ref{cha:methodology} and \ref{cha:implementation}.

    \item \textbf{Copyrights}: OpenWPM simulates \say{real} users and therefore, visits websites using a consumer browser (Firefox). Therefore, no additional information is downloaded and no data is maintained for longer than required and therefore, websites' copyright is respected.

    \item \textbf{Affecting Availability}: OpenWPM visits the websites from the given list only once, gathers the required data and then drops the connection of the website. Furthermore, all additional data analysis will be conducted offline. Therefore, there is no risk of overwhelming websites with traffic and taking the offline (e.g: Denial of Service).
\end{itemize}

\end{document}