\documentclass[../main.tex]{subfiles}

\begin{document}

This chapter discusses in detail the data analysis results. It will analyse their significance, the impact that cookie banner implementations have on internet users and whether they answer the research questions that were raised at the beginning of this project. This chapter can be divided into 2 parts which are summarised in the following list: 

\begin{enumerate}
    \item \textbf{Identifying dark patterns}: To identify whether websites implement their cookie banners in order to steer users towards privacy-intrusive decisions; 
    \item \textbf{Answering the research questions}: To establish whether the results from the cookie banner data analysis answer adequately the research questions set out in the beginning of this project.
\end{enumerate}

\section{Understanding the Results}

\subsubsection{Cookie banners might not be everywhere}
Nowadays, it seems that almost every website presents its users with a cookie notice informing them that the site that they have visited is using Third-Party Cookies. Indeed, this appears to be true since half of the websites in the dataset prompt the user with a cookie banner (Greece 49\%, UK 53\%, see Table \ref{tab:res_cookies_prevalence}). These results are significantly higher than the findings of Eijk et al. \cite{eijk2019impact} who detected a cookie notice on 40\% of the websites that they surveyed. However, their dataset was significantly smaller and used websites from the US as well.

While this project does not make comparisons between the prevalence of cookie banners before and after the GDPR, one may look at the above data and assume that a large number of websites are complying with the EU legislation. This was observed by Degeling et al. \cite{degeling2018we} who found that more websites offered privacy policy pages after the GDPR came into force. However, this might not be entirely true. Specifically, the findings of this project show that 61\% of Greek and 70\% of UK websites store at least one third-party cookie on their user’s browser. As already shown, only 49\% of Greek and 53\% of UK websites display a cookie notice and therefore, suggesting that 15\% of websites in both countries have yet to comply with GDPR (Figure \ref{fig:prevalence_cookie_banners_tps}).

Thus, the initial assumption of an average everyday user is that cookie banners are almost everywhere. On the contrary, it appears that a lot more work is yet to be done. 

\subsubsection{Websites really want users to accept cookies}
It is clear that websites in both Greece and the UK offer approximately 2 privacy options per cookie banner (Table \ref{tab:avg_options}). Furthermore, these options are more likely going to be either only an Affirmative option or a combination of Affirmative and Informational options (Figure \ref{fig:priv_categories_breakdown}). 

Thus, websites either only allow users to accept tracking or go through additional steps to manage their privacy settings or find out more information about how cookies are used and affect their privacy. Therefore, by hiding away information behind additional clicks and making the opt-in options more prevalent, more users will agree to third-party cookie tracking because, simply, there isn’t anything else that they can do. 

This can be considered as \say{Digital Nudging} which is defined by Weinmann, Schneider and vom Brocke as \say{the use of user-interface design elements to guide people’s behaviour in digital choice environments} \cite{weinmann2016digital}. For instance, the more expensive products in supermarket shelves are placed at eye level so that customers are nudged into making unplanned purchases \cite{schneider2018digital}. Similarly, websites always offer Affirmative actions and require multiple clicks for users to opt-out and thus, nudging them towards opting-into tracking.

This has been observed in a large number of websites in the dataset. Therefore, it is affecting a large number of users on a daily basis indicating a dark pattern. Interestingly, this was also observed by the Norwegian Consumer Council which found that large tech companies, such as Google and Facebook, use similar nudging techniques in order to steer users towards privacy-intrusive choices.

\subsubsection{Opting out is hard}
With only a 4.5\% adaptation across Greece and the UK, it is apparent that websites do not implement opt-out buttons on their cookie banners (Table \ref{tab:opt_outs}). Unless they offer an \say{options} button on their notice, there is no way for users to reject tracking. This can be considered a highly privacy-intrusive technique and therefore, a dark pattern. 

Interestingly, the above results contradict previous research on this topic. Habib et al. \cite{habib2019empirical} showed that 89\% of the websites in their sample offered an opt-out option which is the complete opposite to what was found by this project. However, it is worth noting that Habib et al.’s sample consisted of only 150 websites. Furthermore, they only looked at the privacy options offered by CMPs. Therefore, all the cookie banners and their options tend to offer the same privacy options, regardless of the website. 

Thus, it is evident that a larger dataset with a more diverse set of cookie banner implementations shows that direct-opt out options are not prevalent. This is supported by research from Sanchez-Rola et al. \cite{sanchez2019can} They found that reject privacy options can be found in approximately only 4\% of their dataset which consisted of 2,000 websites within the European Union. This dataset and results seem to be matching the findings of this project as well. 

\subsubsection{CTAs are not misleading}
In total, more than 1 thousand unique Call to Actions were identified across Greece and the UK (Table \ref{tab:unique_ctas}). Moreover, the most common privacy category was the Affirmative and second the Informative in both countries (Figures \ref{fig:ctas_gr} \& \ref{fig:ctas_uk}).

While this aligns with the second observation made in this section, the CTAs themselves do not seem to be misleading or ambiguous regardless of how common they might be. For instance, no terms in the Non-Affirmative category were trying to give the illusion of negative consequences if the user chose to opt-out e.g: \say{Opt-out but checkout might not work}. That was consistent across all privacy categories in both Greece and the UK. 

\subsubsection{Have some cookies, they are good for you}
Every cookie banner comes with a short piece of text that usually informs the users of the purpose of that notice. While not very long or extremely specific, it is extremely important since all users will have to read it. 

Specifically, the average length of the cookie banner privacy text is 60 words (Table \ref{tab:privacy_txt_len}). Surprisingly, the most common terms in both countries are almost identical and have similar frequencies (Figure \ref{fig:res_tf_idf}). More specifically, these common terms are \say{experience}, \say{better} and \say{ensure}. Interestingly, terms such as \say{privacy} or \say{tracking} were not found during the data analysis. 

It is clear that cookie banners aim to present the cookies as an instrument that help users enjoy a better experience while using that website. Therefore, users are inclined to click \say{accept}, without understanding the ramifications on their privacy. Even if the cookie banner provides an opt-out option, users might still choose to opt-in to cookies worrying that they will be unable to use that website, yet that might be very far from the truth.

The frequency that this pattern appears in both Greek and UK websites that users in both countries have to deal with such choices, and sometimes misinformation, on a regular basis. Therefore, this can be identified as a dark pattern that websites employ to steer users towards security-intrusive choices.

\subsubsection{Both countries are very similar}
Although both Greece and the UK are in the European Union and therefore, have to adhere to the GDPR they vastly differ in terms of language and size in economy and population. Thus, it can be expected that the cookie banners in each country can be different as well.

However, the findings discussed in this chapter paint a completely different picture. More specifically, the results for one country are almost identical to the other. For instance, the cookie banner privacy text TF-IDF results are almost the same in both the terms as well as statistical frequency. Another, more prominent example, is the distribution of the most common privacy options categories (Table \ref{tab:avg_options}, Figure \ref{fig:priv_categories_breakdown}), where the Affirmative and Informational categories are the most prevalent ones. 

Therefore, it is apparent that the language or the size of a country might not matter on how users experience websites and their cookie banners. This can be attributed to two things:

\begin{enumerate}
    \item \textbf{Mimicking}: One country simply copies the cookie banner implementation of the other. For instance, the UK has a significantly larger e-commerce industry which quickly adapted to the GDPR. Subsequently, Greek e-commerce websites visited their UK counterparts in order to see how they complied with GDPR;
    \item \textbf{CMPs}: Content Management Platforms (CMPs) offer their services across Europe. Therefore, it is possible that Greek and UK websites use the same CMPs and therefore, their cookie notices will be identical. However, this assumption requires further research to be confirmed.
\end{enumerate}

\section{Answering the Research Questions}
In the beginning, this project set out 7 research questions aiming to understand cookie banners and their privacy options better. The following list discusses in detail whether the collected data and their analysis have managed to answer those research questions: 

\begin{itemize}
    \item \textbf{\ref{rq:prevalence}}: Almost half of the websites in Greece (49\%) display a cookie banner while 1,871 (61\%) store Third-Party Cookies. Similarly, there are 6,413 websites (53.7\%) that display a cookie banner, while 8,256 (70\%) of them store TPs on a user’s browser;

    \item \textbf{\ref{rq:options_avg}}: In Greece, the average number of privacy options in a cookie notice is 2.04. Similarly, UK websites offer 1.8 options per cookie banner (Table \ref{tab:avg_options});

    \item \textbf{\ref{rq:direct_opt_out}}: Unfortunately, only 303 Greek websites ($<10\%$) allow their users to directly opt-out from third-party tracking. In the UK, 361 websites (3\%) offer an opt-out option with their cookie notice (Table \ref{tab:privacy_options_categories});

    \item \textbf{\ref{rq:no_options}}: The majority of both Greek and UK websites offer privacy options to their users. Specifically, only 5 Greek websites (0.3\%) offer no options at all. In the UK, 59 websites (0.9\%) do not offer privacy options (Table \ref{tab:no_options_cookie_banners});

    \item \textbf{\ref{rq:common_ctas}}: The Affirmative category is the most common in both countries. Specifically, there are 1,417 (46\%) and 5,635 (48\%) affirmative options in Greece and the UK respectively. Second, comes the Informational category with 752 (Greece, 24.5\%) and 4,405 (UK, 37.7\%) privacy options;

    \item \textbf{\ref{rq:manage_options_count}}: Overall, Greek and UK websites allow users to manage their privacy options. More specifically, 596 Greek websites (19.4\%) offer a \say{Settings} button. In the UK, 1,289 websites (11.03\%) have implemented an \say{Options} button in their cookie banners (Table \ref{tab:avg_options});

    \item \textbf{\ref{rq:common_privacy_txt}}: The average length of the cookie banner text is 66.2 words for Greece and 52 words for the UK (Table \ref{tab:privacy_txt_len}). The 5 most frequent terms in Greece are \say{uses}, \say{experience}, \say{better}, \say{accepting}, \say{website}. For the UK, the 5 most common terms are \say{uses}, \say{best}, \say{ensure}, \say{site}, \say{experience}. The full TF-IDF term results can be found in Appendix \ref{app:data_analysis_spreadsheet}.
\end{itemize}

It is evident that the results adequately answer the 7 research questions. However, it is very likely that the plethora of data collected as part of this survey may be hiding more dark patterns. Thus, the dataset allows for more questions to be asked and further data analysis to be conducted. 

\section{Summary}
This chapter discussed in detail the results from the cookie banner data analysis. More specifically, it looked at how websites implement their cookie banners using dark patterns. For instance, these patterns can be the lack of opt-out options or telling the users that the cookies are only used to improve their experience on the website, which is misleading. Finally, the chapter explored whether the collected data and the results adequately answer the research questions asked at the beginning of this survey. While it found, that these questions have been answered, the results may be hiding more dark patterns and therefore, more questions should be asked.

\end{document}