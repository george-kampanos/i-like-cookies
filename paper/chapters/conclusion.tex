\section{Conclusion}


We set out to conduct the most comprehensive study of cookie banners in the UK and Greece to date in the hope that a more thorough understanding of the cookie banner landscape in the two countries is beneficial for a rage of stakeholders including users, privacy-enhancing technology developers, and policymakers.
% To achieve this, we collected websites from open-source lists such as Tranco and developed purpose-built software to determine whether websites allow crawling by parsing their robots.txt and Terms of Service pages. 
By extending OpenWPM to detect and store cookie banners, over 17,000 websites were crawled and more than 7,000 cookie banners were collected. 

Our results show that although around half of the websites in our dataset display a cookie notice, a substantial proportion do not show one even though they use third-party cookies. 
Furthermore, websites make it extremely difficult for users to opt-out from tracking with only a minority offering a direct opt-out option. 
Our analysis also suggests that websites present cookies as devices that improve browsing experience for the user while the negative aspects of tracking tend to be downplayed. 
Hence, we find clear evidence of websites nudging visitors towards privacy-intrusive choices and violating regulations. 

Although in many cases our results agree with previous studies considering smaller samples, we also found that in some cases, e.g. prevalence of cookie banners and those providing specific options, our observations significantly differ from previous reported values. 
Hence, we hope that our work provides a more holistic view of the landscape of cookie banners in the two countries. 

Future work directions include more comprehensive studies of the cookie banner landscape for other countries (for which our code is available and can be reused), a more detailed analysis and classification of varying cookie banner practices in specific subsets of the dataset, e.g., in different industries, and further analyses of the cookie banner text corpus. 