\documentclass[../main.tex]{subfiles}

\begin{document}

Websites implement cookie banners in order to inform users about the use of Third-Party Cookies as well as allow them to manage their privacy settings. Especially after the GDPR came into force in 2016, more and more websites have added such notice. Therefore, the ubiquitous nature of cookie banners means that they are rapidly becoming part of a user’s daily web browsing experience.

Since users increasingly have to deal with those privacy notices this project aims to provide a better understanding of cookie banners and the privacy options that they provide. More specifically, the research will focus on the types of cookie banners that users have to interact on a daily basis, whether visitors have a choice on whether they are being tracked and how the wording on those cookies notices may affect users’ perception of privacy and tracking. 

Having a comprehensive understanding of the cookie banner landscape is useful not only for users but also for developing technologies that help users manage their privacy rights. Furthermore, policy makers have a better idea about the level of compliance to privacy laws by websites as well as the dark patterns that users encounter on a daily basis.

This project focuses on the cookie banners in Greek as well as UK websites. These countries were chosen for 2 reasons. Firstly, they both adhere to very similar data protection laws, namely the GDPR and the Data Protection Act of 2018 and therefore, it is expected that most websites in these countries will have cookie notices. Secondly, both countries vastly differ in language and population size. Thus, potential differences in how the two populations experience the internet and the cookie banners are going to be highlighted.

\section{Research Questions}

In order to understand cookie banners better and how they affect the users’ internet experience, this project sets 7 research questions. These are summarised in the following list:

\begin{enumerate}[label=\textbf{RQ\arabic*}:, ref=RQ\arabic*, leftmargin=1.65cm]
    \item \label{rq:prevalence} What is the prevalence of cookie banners in popular websites across Greece and the UK? 
    \item \label{rq:options_avg} How many privacy options do cookie banners provide on average?
    \item \label{rq:direct_opt_out} How many cookie banners offer their users a direct \say{opt-out from tracking} option?
    \item \label{rq:no_options} How many cookie banners do not offer any option at all and inform their users that by \say{using this website, they agree to Third-Party Cookies and tracking}?
    \item \label{rq:common_ctas} What is the most common privacy option provided by the cookie banners? 
    \item \label{rq:manage_options_count} How many cookie banners allow their users to manage their privacy settings and control which vendors track them?
    \item \label{rq:common_privacy_txt} What is the average length of the cookie banner privacy text and what are the most common terms that are used to inform users about the use of cookie banners?
\end{enumerate}
    
\section{Contributions}

In order to answer the research questions set above, a number of novel methods and software was developed and a plethora of data was collected throughout the course of this project. All the contributions made are summarised in the following list:

\begin{enumerate}
    \item \label{contr_1} Developed an automated method of scraping and collecting cookie notices on a large scale, using OpenWPM;
    \item \label{contr_2} Conducted a more comprehensive study in Greece and the UK and collected a significantly larger dataset compared to other similar studies such as Habib et al.;
    \item \label{contr_3} Make the tools and data available so that similar research can be undertaken by other researchers in different countries.
\end{enumerate}

\end{document}